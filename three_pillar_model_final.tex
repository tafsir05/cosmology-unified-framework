
\documentclass[aps,prd,twocolumn,superscriptaddress,nofootinbib]{revtex4-2}
\usepackage{amsmath,amssymb,graphicx}
\usepackage[utf8]{inputenc}

\begin{document}

\title{Rethinking the Foundations: A Three-Pillar Model of the Universe}
\author{Tafsir Islam}
\affiliation{Independent Researcher}
\date{\today}

\begin{abstract}
This paper proposes a speculative but conceptually grounded model of the universe in which gravity, dark matter, and dark energy constitute the foundational "pillars" of reality. Traditional physics categorizes the four fundamental forces as the base structure of physical law. However, this theory argues that electromagnetism, the strong nuclear force, and the weak nuclear force are emergent phenomena---"sub-forces"---that arise from deeper interactions and structures tied to these three cosmic pillars. ...
\end{abstract}

\maketitle

\section{Introduction}
Modern physics identifies four fundamental forces: gravity, electromagnetism, the strong nuclear force, and the weak nuclear force. However, key challenges remain in unifying these forces---especially gravity---with quantum mechanics. This paper suggests a new hierarchical model in which gravity, dark matter, and dark energy are not side-effects or anomalies, but the core fabric from which all other forces emerge.

\section{The Three Pillars of the Universe}

\subsection{Gravity}
Following Einstein's general relativity, gravity is understood as the curvature of spacetime caused by mass and energy. However, unlike the other three forces, gravity is geometrical, not mediated by a particle in current models. This indicates it may not be a ``force'' at all in the traditional sense, but a fundamental aspect of spacetime structure.

\subsection{Dark Matter}
Though undetected directly, dark matter's gravitational effects are well-documented through galactic rotation curves, gravitational lensing, and large-scale structure formation. This theory treats dark matter not as a particulate add-on, but as a fundamental agent that stabilizes the architecture of galaxies and cosmic structure.

\subsection{Dark Energy}
Dark energy is the dominant component of the universe, causing its accelerated expansion. It may be an intrinsic property of spacetime, representing an outward pressure counteracting gravitational collapse. Rather than being an anomaly, it may reflect the repulsive aspect of a deeper cosmic balance.

\section{Sub-Forces as Emergent Phenomena}

\subsection{Electromagnetism}
In this model, electromagnetism may emerge from the dynamics of curved spacetime in regions where charged particle interactions become stable. This could be analogous to the way waves arise from fluid dynamics.

\subsection{Strong and Weak Nuclear Forces}
These forces, confined to the atomic scale, may represent localized distortions or phase interactions in the gravitational-dark matter-energy field. They are essential for matter structure, but not necessarily fundamental in the cosmic sense.

\section{Philosophical and Theoretical Implications}
This model aligns with emergent gravity, holographic universe theories, and aspects of string theory. It suggests that the fabric of the universe is built not from particles and fields alone, but from deep interactions between cosmic-scale entities that shape the emergence of all observable physics.

\section{Future Directions and Predictions}
\begin{itemize}
  \item Development of mathematical models linking the ``three pillars'' to quantum field effects.
  \item Reinterpreting gravitational lensing and cosmic microwave background fluctuations through this lens.
  \item Examining the potential geometric structure that unites these pillars at the Planck scale.
\end{itemize}

\section{Conclusion}
The proposal that gravity, dark matter, and dark energy are the primary foundations of physical reality challenges the traditional force hierarchy. If correct, this model may point toward a deeper Theory of Everything---one where what we currently call ``forces'' are emergent behaviors of deeper cosmological structures.

\end{document}
